\documentclass[UTF8]{article}
\usepackage{abstract}
\usepackage{hyperref}
\usepackage[round]{natbib}
\usepackage[english]{babel}
\usepackage[utf8x]{inputenc}
\usepackage{indentfirst}
\usepackage{dashrule}
\usepackage{makecell}
\usepackage{bm}
\usepackage{tabu}
\usepackage{multirow}
\usepackage{booktabs}
\usepackage{esdiff}
\usepackage[flushleft]{threeparttable}
\usepackage{picinpar}
\usepackage{amsmath}
\usepackage{amsthm}
\usepackage{mathrsfs}
\usepackage{amsfonts,amssymb}
\usepackage{algorithm}
\usepackage{algorithmicx}
\usepackage{algpseudocode}
\usepackage{upgreek}
\usepackage{geometry}
\usepackage{setspace}
\usepackage{titlesec}
\usepackage{float}
\usepackage{graphicx}
\usepackage{subfigure}
\usepackage{diagbox}
\usepackage{caption}
\usepackage{lineno}
\usepackage{appendix}
\usepackage{siunitx}
\usepackage{longtable}

\allowdisplaybreaks[4]
\renewcommand{\abstractnamefont}{\Large\bfseries}
\geometry{a4paper,left=2cm,right=2cm,top=2cm,bottom=2cm}
\fontsize{12pt}{\baselineskip}
\linespread{1.5}
\selectfont
\hypersetup{colorlinks=true, linkcolor=blue, filecolor=blue, urlcolor=blue, citecolor=cyan, }

\titleformat*{\section}{\large\bfseries}
\titleformat*{\subsection}{\normalsize\bfseries}
\titleformat*{\subsubsection}{\normalsize\bfseries}

\newtheorem{assumption}{Assumption}
\newtheorem{proposition}{Proposition}
\newtheorem{corollary}{Corollary}
\newtheorem{definition}{Definition}
\newtheorem{remark}{Remark}
\newtheorem{property}{Property}
\newtheorem{lemma}{Lemma}

\newcommand{\vect}{\mathbf}
\newcommand{\set}{\mathcal}
\newcommand{\expect}{\mathbb{E}}
\newcommand{\indicator}{\mathbb{I}}

\floatname{algorithm}{Algorithm}
\renewcommand{\algorithmicrequire}{\textbf{Input:}}
\renewcommand{\algorithmicensure}{\textbf{Output:}}

\title{\textbf{\Large Benders Decomposition}}
\author{}
\date{}

\begin{document}

\maketitle

$\set{R}$: the set of real numbers.

$\set{R}_{+}$: the set of nonnegative real numbers.

$\set{R}_{+}^{n}$: the set of $n$-dimensional nonnegative real vectors.

$\set{Z}$: the set of integers.

$\set{Z}_{+}$: the set of nonnegative integers.

$\set{Z}_{+}^{n}$: the set of $n$-dimensional nonnegative integer vectors.

$\vect{c} \in \set{R}^{m}$, $\vect{d} \in \set{R}^{n}$, $\vect{f} \in \set{R}^{p}$, $\vect{g} \in \set{R}^{q}$, $\vect{A} \in \set{R}^{p \times m}$, $\vect{B} \in \set{R}^{p \times n}$, $\vect{E} \in \set{R}^{q \times m}$.

The original mixed integer linear programming model is formulated as follows.
\begin{align}
	(\text{\rm P1}) \qquad z_{\rm P1} = \min \vect{c}' \vect{x} + \vect{d}' \vect{y}, & \\
	\text{\rm s.t.} \qquad \vect{A} \vect{x} + \vect{B} \vect{y} \ge \vect{f}, & \\
	\vect{E} \vect{x} \ge \vect{g}, & \\
	\vect{x} \in \set{Z}_{+}^{m}, & \\
	\vect{y} \in \set{R}_{+}^{n}. &
\end{align}

Given the value of $\vect{x}$, the optimal objective value associated with variable $\vect{y}$ can be described as follows.
\begin{align}
	(\text{\rm P2}(\vect{x})) \qquad z_{{\rm P2}(\vect{x})} = \min \vect{d}' \vect{y}, & \\
	\text{\rm s.t.} \qquad \vect{B} \vect{y} \ge \vect{f} - \vect{A} \vect{x}, & \\
	\vect{y} \in \set{R}_{+}^{n}, &
\end{align}
where $z_{{\rm P2}(\vect{x})} = + \infty$ if model P2($\vect{x}$) is infeasible.

The dual of model P2($\vect{x}$) is as follows.
\begin{align}
	(\text{\rm D2}(\vect{x})) \qquad z_{{\rm D2}(\vect{x})} = \max \bm{\pi}' (\vect{f} - \vect{A} \vect{x}), & \\
	\text{\rm s.t.} \qquad \bm{\pi}' \vect{B} \le \vect{d}', & \\
	\bm{\pi} \in \set{R}_{+}^{p}. &
\end{align}

Note that the feasible set $\Pi = \{\bm{\pi} \in \set{R}_{+}^{p}: \bm{\pi}' \vect{B} \le \vect{d}'\}$ is independent of $\vect{x}$. If $\Pi$ is empty, then D2($\vect{x}$) is infeasible for all $\vect{x} \in \set{Z}_{+}^{m}$. Because the infeasibility of D2($\vect{x}$) implies that either (i) P2($\vect{x}$) is infeasible, or (ii) $z_{{\rm P2}(\vect{x})} = - \infty$, we conclude that the original problem P1 is either infeasible or unbounded.

In the following, we assume that the original problem P1 is feasible with a finite optimal objective value, i.e., $\Pi$ is nonempty. Furthermore, denote $\{\bm{\pi}_i: i \in \set{I}\}$ as the set of extreme points of $\Pi$, and $\{\bm{\pi}_j: j \in \set{J}\}$ as a complete set of extreme rays of $\Pi$. Then we have 
\begin{itemize}
	\item P2($\vect{x}$) is feasible if and only if $\bm{\pi}_{j}^{'} (\vect{f} - \vect{A} \vect{x}) \le 0$, $\forall~ j \in \set{J}$.
	\item If P2($\vect{x}$) is feasible with finite optimal objective value, then $z_{{\rm P2}(\vect{x})} = z_{{\rm D2}(\vect{x})} = \max_{i \in \set{I}} \bm{\pi}_{i}^{'} (\vect{f} - \vect{A} \vect{x})$.
\end{itemize}

Therefore, 
\begin{align}
	z_{\rm P1} = \min \left\{\vect{c}' \vect{x} + z_{{\rm P2}(\vect{x})}: \vect{E} \vect{x} \ge \vect{g}, \vect{x} \in \set{Z}_{+}^{m}\right\} = \min \vect{c}' \vect{x} + \eta, & \\
	\text{\rm s.t.} \qquad \vect{E} \vect{x} \ge \vect{g}, & \\
	\eta \ge \bm{\pi}_{i}^{'} (\vect{f} - \vect{A} \vect{x}), & \qquad \forall~ i \in \set{I}, \\
	\bm{\pi}_{j}^{'} (\vect{f} - \vect{A} \vect{x}) \le 0, & \qquad \forall~ j \in \set{J}, \\
	\vect{x} \in \set{Z}_{+}^{m}. & 
\end{align}




\bibliographystyle{plainnat}

\end{document}
